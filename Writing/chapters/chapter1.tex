\begin{savequote}[75mm] 
One  of  the  salient  features  of  the  world  economy  has  been  the  important  surge  in  trade  and 
financial globalization in the last two decades. Multiple free trade agreements and regional integration 
agreements  are  being  celebrated  —with  more  than  400  regional  trade  agreements  in  force  by 
December 2008 according to WTO/GATT. In addition, world trade grew at least twice as fast as world 
output over the last two decades, thus deepening economic integration. 
\qauthor{C\'{e}sar Calder\'{o}n \& Virginia Poggio} 
\end{savequote}

\chapter{Introduction}
\label{sec:Introduction}

\newthought{Trade liberalization,} particularly the removal of import barriers such as tariffs or quotas, 
provides higher access for domestic firms and consumers to purchase goods in international markets. 
Consumers gain access to a larger, and perhaps higher quality, variety of goods. 
Domestic firms find it easier to import intermediate goods in their production process.
Such firms may have a ``love of variety'', and are more productive when using 
a varied bundle of intermediate inputs \citep{dixit1977monopolistic}. On the other hand, 
inefficient firms that produce goods that compute with imports domestically 
may be hurt by trade, even as the country as a whole gains from trade, by shifting entrepreneurial
activity to more productive uses \citep{holmes2}, or other, well established mechanisms.

As some firms benefit and some firms are hurt by opening up to trade, a natural question arises:
what is the effect of trade liberalization upon the wages of workers in domestic firms?  
This is one of the most important questions in international trade, and has generated
an extensive literature\footnote{See \citet{feenstraglobal} and \citet{goldberg} for a review
of the recent literature regarding trade liberalization and wages, particularly in the context
of developing countries.}. The contribution of this paper 
is to empirically examine and quantify the impact of trade liberalization upon workers in a
developing country context, particularly the Dominican Republic. Additionally, I examine 
the heterogeneous impacts of trade based upon the skill level of workers, and test a key implication of
the Heckscher-Ohlin model of international trade. 

I test the impact of trade liberalization in the context of the Central American Free Trade Agreement
(CAFTA, or CAFTA-DR), which primarily lowered input tariffs in the Dominican Republic on goods imported from
the United States and other countries in Central America. One peculiarity of this agreement is
that it left output tariffs largely unaffected, which allows me to focus solely on the impact 
of input tariffs. Ultimately, CAFTA reduced average tariff rates from 12.06\% to 2.73\%, a similar 
nominal decrease to the reduction in tariffs in Mexico as a result of NAFTA. I construct a panel
dataset of labor market outcomes in 2002 and 2013 in the Dominican Republic, which corresponds to
one sample 5 years prior to the implementation of CAFTA and six years after. I find that tariffs
remained largely constant from 2002-2007, so the change in tariff rates from 2002-2013 can be
almost entirely attributed solely to the trade agreement. Furthermore, having a sample six years
after the trade agreement allows for firms to fully adjust their composition of inputs and make hiring, firing,
and wage decisions according to changes in trade barriers, if one believes there may be a lag 
in these changes post-CAFTA.

I examine the impact of trade reform at the municipality level, and at the level of each occupation
within a municipality, with the expectation that I will find clearer effects of tariff changes at
the occupational level. At the municipality level, I find insignificant, and somewhat mixed depending
on the specification, effects of trade liberalization. However, at the occupational level, 
I find that a 10 percentage point decrease in the change in input tariffs is associated with 4.5 
percentage point lower wage growth over the period 2002 to 2013. As supporting evidence for this result, I find that
workers in occupations in the nontraded sector experienced faster wage growth during the period 
(about $1.6$ percentage points faster) vis-\`{a}-vis workers in occupations categorized as competing with 
imports. Finally, upon estimating my occupational level results on a high and low skill subsample, 
I find  that the wages of skilled workers experienced slower wage growth than their unskilled 
counterparts from 2002 to 2013. Since I argue that the Dominican Republic is relatively unskilled labor
abundant, this is in line with theoretical predictions of the Heckscher-Ohlin Model 
for a developing country.

Narrowly, my work helps to estimate the effect of CAFTA on the wages of workers in the 
Dominican Republic, but the results have broader implications. Specifically, 
I provide further evidence on who gains the most from the removal of trade barriers in developing
countries, particularly unskilled workers and laborers not employed by import competing firms,
and quantify this effect. Due to the peculiar nature
of the Central American Free Trade Agreement, I am able to disentangle the effects
of reducing trade barriers on input goods from the resulting effects of lowering tariffs on outputs.
This helps to clarify the channel through which trade liberalization impacts wages, particularly
the significant effect of lowering input tariffs (to the contrary, \citet{amiti2012trade} and others 
find that lowering output tariffs has insignificant effects). 
%To my knowledge, there are only a few other studies which have examined the direct effects of lowered 
%input  tariffs on labor market outcomes, absent from changes in output tariffs\footnote{\citet{amiti2012trade}
%and \citet{amiti} being some of the few examples to consider the impact of reducing input 
%tariffs on wages.}. 
Ultimately, this paper adds to the body of work examining the direct effects of lowered 
input  tariffs on labor market outcomes.

%The organization of the rest of the paper is as follows; 
%I introduce the context of CAFTA in Section \ref{sec:Context},
%present the theory and relevant literature in Section \ref{sec:Litreview},
%elaborate upon whether CAFTA-DR presents a natural experiment in Section \ref{sec:Exogeneity},
%overview the construction of my data set in Section \ref{sec:Data}
%and discuss implications of the model
%in Section \ref{sec:Results}. Section \ref{sec:Conclusion} concludes.