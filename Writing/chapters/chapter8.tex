\chapter{Conclusion}
\label{sec:Conclusion}

\newthought{One of the} most important questions in international trade is the effect of trade liberalization
upon domestic firms and the wages paid to their workers. In this paper, I quantify the impact of 
the Central American Free Trade Agreement (CAFTA-DR) upon the wages of workers in the Dominican 
Republic at various levels of aggregation. The trade agreement has exploitable peculiarities, 
particularly that it lowered input tariffs on imports from member countries (including the United States) 
into the Dominican Republic but left export tariffs largely unchanged. Furthermore, I argue that 
the trade liberalization largely occurred uniformly, which minimizes concerns that tariff changes
are endogenous to prior industry performance. Using panel estimation, I examine the effect
of changes in tariffs from 2002 to 2013 on worker wage rates during the same period at both the 
municipality level and at the level of each occupation within a municipality. 
Since CAFTA was implemented in 2007, this panel allows for local labor markets to fully adjust 
to new changes in tariff barriers, and since tariffs remain largely constant between 2002 and 2006, 
changes in duties between 2002 and 2013 are mainly due to the free trade agreement, and not other 
sources. Additionally, I examine the heterogeneous impacts of trade based upon the skill level of 
workers, and test a key implication of the Heckscher-Ohlin model of international trade. 

Ultimately, at the municipality level, I generally find no statistically significant effects of 
trade reform on the average wages of workers within a municipality. Since at this level I am examining
the effect of an average tariff, weighted by the share of firms in each sector within a municipality,
on average wages, it is natural to expect that this relationship will not be well established.
However, at the occupational level, I find that a 10 percentage point decrease in the change in input 
tariffs during the period is associated with 4.5 percentage point lower wage growth during the study 
period. To corroborate these I findings, I find workers in occupations in the nontraded sector experienced 
faster wage growth during the period than their counterparts in the traded sector. Finally, upon 
duplicating my estimating equations on both a high and low skill subsample, 
I find  that the wages of skilled workers experienced slower wage growth than their unskilled 
counterparts from 2002 to 2013. If the Dominican Republic is relatively unskilled labor
abundant (I present evidence that this is indeed the case), this is in line with theoretical predictions 
of the Heckscher-Ohlin Model for a developing country.

In this paper, I provide some of the first estimates of the effect of the Central American
Free Trade Agreement on labor markets in the Dominican Republic. As free trade has become 
a contentious political issue, quantifying the effects, and winners and losers, of recent trade 
agreements is important in and of itself. However, due to the structure of the free trade
agreement I am able to examine one particular channel through which trade reform affects wages,
and contribute to the body of work focusing on the heterogeneous
effects of import tariffs on wages. 