\chapter{Estimation Strategy}
\label{sec:Model}

\newthought{In many cases}, the method of testing an empirical relationship between several economic variables must 
be derived from a theoretical model in order to be credible and avoid reporting spurious correlations.
In the case of determining the effect of trade liberalization on regional wages, however, the theoretical
relationship is well established, either by the Heckscher-Ohlin model,
the specific-factors model of regional economies, the model presented by \citeauthor{amiti}, or 
others. I remain agnostic between these models, but it is clear the relationship between tariff rates
and worker wages is well established. Ultimately, I adopt an estimation strategy to examine the 
effect of changes in tariff rates on wages that bears similarities to each of \citet{kovak}, 
\citet{amiti}, and \citet{amiti2012trade}. 

In a given year, the expected reduced form relationship between wages and the tariff rate at the either
the municipality level $(m)$ (or occupational level $(o)$, within each municipality) is given by 
\begin{equation}
\label{eq:Equation1}
\log(w_m) = \delta_0 \iota + \delta_1 \log(\bar{t}_{m}+1)+Y_m \Gamma + Z_m \Theta + \epsilon_m,
\end{equation}
where $w_m$ is the average wage rate in a municipality, 
I have $\iota = (1,\dots,1)^{\top}$ so I include a constant term $\delta_0$, 
$\bar{t}_{m}$ is the average municipality level tariff rate (see section \ref{sec:Data} for details on 
the construction of this variable for each estimation)\footnote{I include $1$ in the $\log(t+1)$
to make sure no values are $-\infty$.}, $Y_m$ is a matrix
formed by time-invariant controls (such as geographic characteristics of municipalities), and 
$Z_m$ is a matrix formed by time varying characteristics (such as average education levels, 
measures of firm concentration within a region, or time varying geographic fixed effects).
Here, $\beta_1$ is my coefficient of interest, $\epsilon_m$ are my municipality level 
disturbances, and $\log(\cdot)$ is the natural logarithm. As mentioned, I add geographic
\footnote{Either municipality level or province level fixed effects,
where province is one administrative level above the municipality level in the Dominican Republic.} 
fixed effects to some of my equations. The rationale for this is that there may be confounding 
time-varying macroeconomic shocks which effect the wages of workers at the municipality level, outside
of my variables of interest. To the degree that these macroeconomic shocks affect the
wages of workers, we would expect them to affect wages at the national (or perhaps provincial) level,
but have few heterogeneous effects at the regional level. These time varying, national shocks 
should be absorbed by regional fixed effects.

However, I am interested in the effect of the change in tariffs due to CAFTA on the changes in wages
in the Dominican Republic, and so I consider the following long differenced equation:
\begin{equation}
\label{eq:Equation2}
\Delta\log(w_m) = \beta_0 \iota + \beta_1 \Delta\log(\bar{t}_{m}+1)+ \Delta Z_m \Lambda + \nu_m.
\end{equation}

Here, $\Delta \log(w_m) := \log(w_{m,2013}) - \log(w_{m,2002})$,
$\Delta\log(\bar{t}_{m}+1) := \log(\bar{t}_{m,2013}+1)-\log(\bar{t}_{m,2002}+1)$, 
and $\Delta Z_m :=  (Z_{m, 2013}-Z_{m, 2002})$. As this setup is somewhat complex,
I provide some interpretation on how to read coefficient estimates of this relationship.
It helps to note that using logarithm rules, the long differences can be rewritten as 
$\Delta \log(w_m) = \log \left(\dfrac{w_{m,2013}}{w_{m,2002}} \right)$ and likewise for tariffs.
From this, the coefficient of interest can be interpreted as follows: a $1$ percentage point increase 
in $(\bar{t}_{m,2013}+1)/(\bar{t}_{m,2002}+1)$, which corresponds to a \textit{smaller} decrease in 
tariffs between 2002 and 2013, is associated with a $\beta_1$ percentage point larger increase in wages
between 2002 and 2013, all else equal. Inversely, a $1$ percentage point larger tariff decline is associated
with a $\beta_1$ percentage point smaller increase (or larger decrease) in wages 
between 2002 and 2013, which is my preferred interpretation.

I now expound on the expected sign for the coefficient of interest. \text{Ex ante}, the sign of $\beta_1$ at the 
municipality level is somewhat ambiguous -- one may expect that workers in inefficient import-competing 
firms will be hurt by trade liberalization, whereas workers in firms with a ``love of variety'' may 
benefit \citep{dixit1977monopolistic}. However, in a similar empirical study and reduced form relationship,
\citet{kovak} finds municipalities in Brazil that experienced 10 percentage point larger trade 
liberalization experienced roughly 4 percentage point larger wage declines over a 9 year period.
If the context of CAFTA-DR and the Dominican Republic is similar, we might expect that $\beta_1 > 0$.
Ultimately, I expect that if there is a channel between decreases in input tariffs and changes in
wages, it will appear more strongly at the occupational level\footnote{And correspondingly,
even more strongly at the firm level, although I do not have data available at this level.} 
than the municipality level.

The advantages of such a panel estimation are numerous. First, the long differencing helps 
wash out measurement error and any problems with unit roots that may appear in a levels equation 
\citep{amiti2012trade}.
Second, the time-invariant controls $Y_m$ (many of which I do not have data for) are wiped away,
leaving only time varying characteristics and region-year specific fixed effects, which contain
information on exogenous shocks to wages.

That said, there are several challenges to this estimation strategy. The first is that 
tariff changes may have been limited to only certain industries with insufficient political
capital to lobby against them, and so tariff changes reflect endogenous industry performance.
However, I make the case that tariff changes due to CAFTA are arguably exogenous to 
firm and industry performance in Section \ref{sec:Exogeneity}. Furthermore, even if tariff
reform is not politically exogenous, and if political 
economy factors relevant to tariff negotiations are time-invariant, then using long differencing
\ref{eq:Equation2} would wipe those factors away.

The second concern is that changes in the wage rate from 2002 to 2013 may be a reflection of changes 
in supply, and not of changes in trade-induced demand. That is, changes in tariffs may lead workers,
who are fairly mobile, to migrate within the Dominican Republic towards municipalities with higher 
average protective input tariffs. In general equilibrium, wages would change accordingly to this shift 
in supply. To get a sense of demographic and labor market changes in comparison to tariff changes across
municipalities in the Dominican Republic, figures \ref{fig:Map4} to \ref{fig:Map3} plot average
tariff rates and population statistics. Note that there are large migrations of people from the rural
areas to large cities (such as southern Santo Domingo) during this time. 
To establish whether or not there is an association between tariff changes and migration,
I estimate the reduced form relationship

\begin{equation}
\label{eq:Equation3}
\Delta\log(\text{Outcome of Interest}_m) = \alpha_0 \iota + \alpha_1 \Delta\log(\bar{t}_{m}+1)+\eta_m.
\end{equation}

The results of this estimation are displayed in Figure \ref{fig:Table3}; I estimate whether there
is an association at the municipality level between the change in average tariff rates and the change in 
population, change in total size of the workforce, and change in the employment rate. I find that
(regardless of the inclusion of province level fixed effects or not), changes in tariffs have an
insignificant effect on these outcomes. This, and the prior point, lead me to conclude that my 
estimation strategy will produce unbiased estimates for the effect of trade reform.

For my regressions at the occupation level, I adopt a reduced form relationship similar to that
of Equation \ref{eq:Equation2}, but where wages are averages at the ISIC 2 digit occupational level
within a given municipality, and tariffs are converted to occupational averages at the ISIC 2 digit level. 
Here, my reduced form relationship is given by

\begin{equation}
\label{eq:Equation4}
\Delta\log(w_{m,o}) = \beta_0 \iota + \beta_1 \Delta\log(\bar{t}_{o}+1)+ \Delta Z_{m,o} \Lambda
+X_{m,o} \Omega+\upsilon_{m,o}.
\end{equation}

In addition to the explanatory variables in Equation \ref{eq:Equation2}, I include additional variables 
that are measured as levels in either 2002, 2010, and 2013 in the matrix $X_{m,o}$. 
These variables include measurements of the estimated number of workers in the given occupation
within a municipality, average education levels for a given occupation/municipality pair, and 
following from \citet{holmes1}, a measure of the concentration of an occupation within a municipality.
One challenge at the occupational level is that many of the occupations listed in my survey data do
not have corresponding tariff information (see section \ref{sec:Data} for more details). Consistent 
with the literature, I set the tariff change for these occupations to zero, although in some samples I 
drop them altogether or include a dummy variable to indicate whether an occupation is non-traded.

At the occupational/municipality level, I expect the effect of reduced trade barriers to have a 
much clearer impact on the wages of workers than at the municipality level. That said, 
\textit{ex-ante} the sign of the coefficient of interest may be ambiguous. Following from the 
theory presented by \citet{amiti}, there may be heterogeneous effects of trade reform on workers,
depending on the characteristics of their respective firms. Upon lowering trade barriers, firms 
that import intermediate inputs may raise wages relative to non-importing firms, or firms that
compete with inputs. As my data does not distinguish firms based on these characteristics, the sign
of $\beta_1$ may be hard to predict. However, if higher trade barriers were initially enacted to
protect firms that compete with inputs, then I may expect that $\beta_1>0$. In other words, I expect
that, on average, occupations facing larger tariff decreases were concentrated in industries that
competed with imports to begin with. If this is the case, then upon inclusion of a dummy variable
to indicate whether an occupation is classified as being in the non-traded sector (where
the dummy equals one if the occupation does not produce traded goods), that coefficient
should be negative (in general, I expect this coefficient to have the opposite sign of $\beta_1$).
For the coefficient corresponding to occupation concentration, the expected sign comes from
\citet{holmes1}. The authors' model suggests that adverse wage impacts due to import competition 
should be exaggerated in regions where one industry is more concentrated, so I expect that
this coefficient is negative. I only include a measure of baseline occupation concentration in 2002,
as one might expect that firm concentration after the implementation of CAFTA is endogenous to 
tariff rates.

\textit{Heterogeneity based on Education Levels}

Consistent with the predictions of the Heckscher-Ohlin Model, I wish to test whether opening to
trade reduces the wages of skilled laborers relative to unskilled laborers, as unskilled labor
is relatively abundant in the Dominican Republic\footnote{To see this, refer to Figures \ref{fig:Summary 1}
and \ref{fig:Summary 2}, which shows that on average, education levels in the Dominican Republic,
as measured in years of education, are less than high school level. To the degree that education
is a reliable proxy for skill level (I believe this is a reasonable assumption), this implies that 
the Dominican Republic is relatively abundant in unskilled labor.}. 
As such, I wish to test whether there are varying effects of trade reform on wages for high and 
low skilled workers. To test this, I subset my full dataset into two samples: one where all
the respondents have only a primary-level education or below (0-8 years of education), or
a secondary level education or higher. Although these subsamples do not fully capture whether a respondent
performs a task that is high or low skilled for employment, on average I expect this distinction to
reflect skilled or unskilled employment. Once I have split up the samples based on education levels,
I then repeat all the same calculations as in the full sample. From here, I test the reduced form 
relationship given in Equation \ref{eq:Equation4} for each of these subsamples. 