\chapter{Exogeneity of Tariff Decreases}
\label{sec:Exogeneity}

\newthought{Before} discussing the impact of tariff changes on labor markets in
the Dominican Republic, it is first necessary to comment upon the political economy of
tariff negotiations. \citet{grossman}, \citet{brock1978}, \citet{maggi2007political} and others, 
have observed the potential for special interest lobbies within a country to have a significant 
impact upon policymakers and their decisions, particularly when deciding which trade barriers to 
remove and which to leave in place. One consequence of this is that 
tariff rates and tariff rate reductions could be determined by factors endogenous to firm level
wages; tariffs can be viewed as the result of a political process, which may be intertwined with 
various aspects of the performance of regional labor markets. If this is the case, then estimates 
for the effect of tariffs on labor market outcomes will suffer from omitted variable bias, if not
corrected for. Therefore, I discuss the qualitative and
quantitative evidence available on the potential exogeneity of the initial level of
trade barriers and tariff rate reductions as a result of CAFTA. 

First, note that the level of Dominican trade protection in 2002 bears a large resemblance
to tariff barriers in place in 1996\footnote{Detailed
information on trade barriers is not available further back than this} (see Figure \ref{fig:Graph4}).
Many of these tariff bariers were set decades ago, potentially as the result of a political process
that took place twenty years or more prior to CAFTA. Therefore, it is possible that pre-CAFTA level 
duties on many goods were reflective of prior bargaining, not of modern political processes.
If one assumes there may exist institutional constraints preventing lowering tariffs without an 
intervention from another country (the United States, in this case), then the pre-CAFTA tariff
level may be able to be considered an arbitrary result of a historical process. 

Although Dominican policymakers negotiated tariff rate reductions bilaterally with the
United States, regardless, the goal of the agreement was to lower tariffs on all
incoming goods. Detailed accounts of the tariff negotiation process suggest that Dominican policymakers,
in addition to Dominican Presidents Hip\'{o}lito Mej\'{\i}a and Leonel Fern\'{a}ndez\footnote{Mej\'{\i}a
serving as president of the Dominican Republic from 2000-2004, Fern\'{a}ndez serving from 2004-2012.},
were in favor of achieving broad reductions in input tariffs through CAFTA. Sonia Guzm\'{a}n de 
\citeauthor{guzman}, 
the head negotiator in the CAFTA-DR process for the Dominican Republic, writes that the Dominican 
government estimated that ``over 300,000 local jobs depended
on the commercial exchange between the two countries'', and that these workers stood to benefit from the 
free trade agreement. Certainly, there were certain local industries, such as shrimp producers,
which lobbied against lower tariff rates. However, Guzm\'{a}n states that Regina Vargo, an 
assistant U.S. trade representative, claimed that these were political non-starters in the negotiation;
emphasizing that there were certain ``things you can't say no to''. Ultimately,
negotiators reached an agreement where trade barriers on 99.5\% of all Dominican products sold to 
the U.S. and 78\% of all U.S. goods to sold to the Dominican Republic would be removed entirely by 2012
(if they weren't already duty-free, as was the case for many Dominican exports) \citep{guzman}. 
After the agreement was adopted and ratified, the United States
forbade the Dominican Congress from making any further modifications to the agreement, 
minimizing concerns of political influences that could have taken place after the bilateral
negotiation process had ended \citep{usambassador}.

As quantitative evidence suggestive of the uniformity of tariff reductions, 
I compare the relationship between pre-CAFTA tariff
levels and the amount of tariff reduction in figures \ref{fig:Graph1}, \ref{fig:Graph2} and 
\ref{fig:Graph3}. If policymakers had the interest
in lowering tariffs uniformly, one would expect to see larger tariff reductions on products that
initially had higher protection levels. Indeed, for average tariffs on Harmonized System 
6 digit product codes (Figure \ref{fig:Graph1}), there is a linear ($R^2=0.85$) relationship 
between the initial amount of protection for a good and
the amount of tariff decrease. These qualitative and quantitative facts
suggest that Dominican input tariffs
were lowered (more or less) uniformly as a result of CAFTA-DR, as desired.