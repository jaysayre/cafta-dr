\chapter{Literature Review}
\label{sec:Litreview}

\newthought{There is} a considerable existing literature discussing the link between trade openness
and changes in the wages of workers.
The canonical model of international trade, the Heckscher-Ohlin (H-O) model, 
predicts that a country abundant in a given factor will specialize 
in the production of goods that use that factor relatively intensively.
In the case of a developing country such as the Dominican Republic, the 
simple 2$\times$2 H-O model (where the factors are skilled and unskilled labor) suggests that the 
country will specialize in the production of goods
which use unskilled labor relatively intensively in relation to skilled labor. 
The connection to the income distribution of a nation comes from the related Stolper-Samuelson Theorem,
which asserts that as a nation moves from autarky to free trade, the owners of the relatively
abundant factor, such as unskilled labor, will find their real incomes rising, while the owners of
the relatively scarce factor will find their real incomes falling. 
Therefore, upon lowering trade barriers, and thus increasing the price of unskilled labor intensive
good, unskilled workers may be expected to see their wages
increase. However, empirically validating the Stolper-Samuelson is difficult for several reasons. 
The Stolper-Samuelson Theorem refers to economy-wide factor returns \citep{goldberg}, not the 
incomes of workers in a  given intra-country region. Next, pre-existing data on the  proportions
of various factors is not easily available for many developing countries, such as the Dominican Republic.
Finally, establishing a firm link between tariffs and wages may be difficult in the presence of external
macroeconomic shocks. That said, even absent this data, in a Heckscher-Ohlin framework, trade 
liberalization will be generally expected to reduce the premium for skilled labor in middle-low 
income or developing countries. 

As part of an extensive literature, several papers that examine the effect of trade liberalization on 
wages, particularly the wage premium, are Pavcnik, Blom, Goldberg, and Schady (2004), Goldberg
and Pavcnik (2004), Mishra and Kumar (2005), Feliciano (2001), and Kaplan and Verhoogen (2005).
These papers find mixed results, some find positive associations between trade reform and wages 
(Goldberg and Pavcnik (2004)), others find negative associations (Mishra and Kumar (2005)), and some 
find no relationship (Pavcnik, Blom, Goldberg, and Schady (2004), Feliciano (2001)). As noted by 
\citet{goldberg}, ``the heterogeneity of findings in these studies is perhaps not surprising 
given the large number of possible channels through which trade could affect industry [wages 
and wage premia]''. Therefore, I review some of the literature discussing such channels through
which trade reform can affect wages.

One paper which introduces a model of trade liberalization and its effect upon wages is \citet{amiti},
which finds varying impacts of reduced trade barriers based upon the characteristics of
firms. Their model suggests that a decline in input tariffs raises the wages of workers at firms
using imported inputs, but reduces wages at firms that do not import inputs. 
\citeauthor{amiti} find that a 10\% point fall in input tariffs has an insignificant impact
on wages in firms that do not import but increases wages in firms that do import. 
To replicate these findings, it is necessary to obtain plant-level information on workers'
wages, and to determine the composition of inputs into the production process for each firm.
My data set, however, does not allow me to link workers to their respective firms, and thus only
provides information regarding the average wages of workers in a given occupation. 
In related work, \citet{amiti2012trade} examine the decline of input tariffs on the wage skill
premium of workers in Indonesia, a country with a large share of unskilled labor.
They find that a 10 percentage point decrease in input tariffs reduces the skilled wage premium
by 10 percent for firms that import, consistent with the predictions of the H-O model, and suggestive
of firms substituting imported inputs for skilled production of those inputs. Looking separately at
output tariffs, \citeauthor{amiti2012trade} find no statistically significant impact on the 
skill premium within firms from changes in barriers applied to output goods.

Another recent paper which explores the heterogeneous impacts of trade upon domestic firms is 
\citet{holmes1}, which develops a model for how international trade affects domestic plants of varying sizes. 
Examining the effects of a surge in Chinese manufactured goods on U.S. manufacturers,
\citeauthor{holmes1} find that import competing plants that were large or produced standardized goods 
either closed down or laid off many of their workers, while smaller firms that produced specialized 
goods fared better, even \textit{within} a given industry classification. The authors' 
model suggests that adverse wage impacts due to import competition should be more pronounced in 
areas where one industry is more concentrated. Although the data I have does not provide detailed
on plants and their relative specialization, my firm level data allows one to infer how concentrated an
industry/occupation is within a municipality.

Finally, \citet{kovak}  examines 
the effect of trade liberalization on regional wage changes in Brazil. As a result of long standing
import substituting industrial policies, in 1987 the average tariff level in Brazil was high; 
54.9 percent. However, these were unsustainable, and by 1995, policymakers reduced average tariffs 
to 10.8 percent. \citeauthor{kovak} calculates, for each region, a measure for the share of regional production 
accounted for by each industry, and then for each of these industries estimates the effect tariff 
changes have had upon local wages in a region. 
To estimate these effects using reduced form equations relies upon the exogeneity of tariff changes to 
industry performance; that tariff changes have not been limited to only certain industries. 
The author argues that, in the context of Brazil, policy makers had explicit aims to cut tariffs 
uniformly, without prioritizing one industry over another, which is corroborated by showing that 
tariff cuts were largest in industries that had high barriers to trade initially.
Ultimately, \citeauthor{kovak} finds that a region facing 10 percentage point larger tariff-induced 
price decline experienced a 4.39 percentage point larger wage decline.