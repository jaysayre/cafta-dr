\chapter{Results}
\label{sec:Results}

\newthought{I estimate} the reduced form relationships above, particularly equations \ref{eq:Equation2}
and \ref{eq:Equation4}, using ordinary least squares (OLS) estimation.  At the municipality
level, Figure \ref{fig:Table1} displays the results of the estimation of equation \ref{eq:Equation2}. 
The main coefficients of interest $(\beta_1$'s) are those displayed in the $\Delta\log(t+1)$ row. In the simplest
specification (column 1), I observe that a 10 percentage point decrease in the tariff rate change between
2002 and 2013 is associated with a 2.9 percentage point larger decrease in the wage rate change between 
2002 and 2013, ceteris paribus, which is a statistically significant result. However, upon including 
province level fixed effects (column 2), the point estimate for $\beta_1$ decreases, and is no longer 
statistically significant. In other specifications, the coefficient estimate is mostly statistically 
insignificant, but generally positive (i.e. $\beta_1 > 0$, as expected).

At the occupational level, Figure \ref{fig:Table2} displays the results of the estimation of equation 
\ref{eq:Equation4}. Again, the main coefficients of interest $(\beta_1$'s) are those displayed in the 
$\Delta\log(t+1)$ row. In every specification, the point estimates for $\beta_1$ are all positive
and highly statistically significant at the $1\%$ level. Examining the results in column 3, which
includes municipality level fixed effects, I observe that a 10 percentage point decrease in the tariff 
rate change is associated with a 4.5 percentage point larger decrease in the wage rate change between 
2002 and 2013, ceteris paribus. In column 4, I include in my sample the occupations that are nontraded,
which decreases my point estimate for $\beta_1$ slightly, indicating that workers not exposed to 
the direct effects of trade liberalization experienced larger wage growth during the period. 
Corroborating
this, I include a dummy variable for whether an occupation is nontraded, and find that workers in
occupations not facing competition from imports experienced $.16$ percentage point faster wage growth
during my study period vis-\`{a}-vis workers in the traded sector. In column 6, I include a measure of
occupation/firm concentration within a
municipality, which takes values in the interval $[0,1]$. I find that occupations which are more highly 
clustered within a municipality experience larger wage growth during the period of trade liberalization,
all else equal. This is inconsistent with the 
predictions of \citet{holmes1}, however, this result is statistically insignificant.
In column 7 I interact this measure
of occupation concentration with the initial trade barriers faced by an occupation.
Upon inclusion of this interaction, I still do not obtain the desired negative coefficient on the tariff
change/occupation concentration interaction term, although this is again statistically insignificant.
Surprisingly, including the number of workers in a municipality seems to have no effect on my
results, suggesting that there are no heterogeneous impacts of trade liberalization in larger/smaller
regions of the country.

For my occupational results segregated by the educational level of workers, see Figures \ref{fig:Table4}
and \ref{fig:Table5}. Between the two tables, many of the results remain broadly similar. In both,
the coefficients of interest ($\beta_1$) are positive and statistically significant. For both,
I consider a variety of different specifications for robustness, but the main column of interest
in both is column 3, which includes municipality level fixed effects. In the high skill sample,
I observe that a 10 percentage point decrease in the tariff 
rate change is associated with a 3.6 percentage point larger decrease in the wage rate change between 
2002 and 2013, all else equal. In the low skill sample,
I observe that a 10 percentage point decrease in the tariff 
rate change is associated with a 1.3 percentage point larger decrease (or smaller increase) 
in the wage rate change between 2002 and 2013, all else equal. Therefore, these tables suggest
that relatively skilled workers were more adversely affected by trade liberalization than unskilled
workers, a result consistent with the predictions of the Heckscher-Ohlin model.