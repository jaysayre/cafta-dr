\documentclass[12pt]{article}

\usepackage{setspace}
\usepackage{caption}
\usepackage{graphicx,amsmath,amsfonts,amssymb,amsthm}
\usepackage{rotating,fancyvrb}
\usepackage[svgnames]{xcolor} % Needed for midnight blue color
\usepackage[colorlinks]{hyperref}
\hypersetup{citecolor=MidnightBlue,urlcolor=DarkBlue} %Hyperlinks with different coloring
\usepackage{float}
\usepackage{natbib}
\usepackage{booktabs}
\usepackage[margin=1in]{geometry}
\usepackage{extramarks}
\usepackage{fancyhdr}
\usepackage{indentfirst} % Indent first sentence of a new section.

% Title Page Formatting
\title{
\textmd{\textbf{\Title}}
\textbf{\author{\AuthorName \\ \Class \date{\today}}}}
\normalfont
%\pagestyle{fancy} %for header with section title

\begin{document}

\newcommand{\Title}{Trade Liberalization and the Dominican Republic}
\newcommand{\Class}{Economics 4993}
\newcommand{\AuthorName}{James Sayre}
      
\maketitle
\doublespacing
\setcounter{footnote}{0}
\renewcommand{\thefootnote}{\arabic{footnote}}

\vspace{-25pt}
\begin{abstract}
Over the last several decades, trade barriers have fallen substantially, and agreements promoting
free trade between countries have proliferated. One such agreement, the Central American Free Trade 
Agreement (CAFTA, or CAFTA-DR), aimed to lower trade barriers between Central American countries and 
the United States.
\end{abstract}



\vspace{-10pt}
\section{Context}
\label{sec:Context}
The Central American Free Trade Agreement was modeled upon the success of the
North American Free Trade Agreement between the United States, Canada, and Mexico (?).


In 2003, negotiations began on the Central American Free Trade Agreement, with Costa Rica, El 
Salvador, Guatemala, Honduras, Nicaragua, and the United States taking part in the discussions. The 
Dominican Republic joined the negotiations in early January, retitling the agreement the Dominican 
Republic-Central America Free Trade Agreement. The goal of the bill was to create a free trade zone 
similar to the one created by the North American Free Trade Agreement in 1994. President George W. 
Bush of the United States signed CAFTA-DR into law in 2005, but it took another two years for the 
Dominican Republic to fully implement the agreement, which it did in 2007.

One of the explicit aims of CAFTA-DR was to reduce tariffs uniformly for United States imports into 
member Central American countries, and to "progressively eliminate customs duties on originating 
goods" \citep{ustraderep}. Tariffs on most products exported to the United States (U.S.) from Central America were 
already duty-free as part of the Caribbean Basin Initiative, and so CAFTA-DR largely eliminated 
tariffs on imported American goods, rather than eliminating American tariffs on exports of Central 
American products. The bill put a moratorium on establishing new tariff lines or raising customs 
duties between the parties involved, and explicitly defined a time line for each good to have its 
tariffs reduced. Based upon World Trade Organization data\footnote{Using a import-weighted average of tariff rates for Harmonized System two-digit product codes, the nominal average decrease is larger}, I find CAFTA-DR reduced average Dominican 
Republic (D.R.) tariff rates on imports from member countries from 12.33\% to 2.73\% from 2006 to 
2007. Many goods were to be declared duty free initially upon implementation of the agreement, but 
many more were to have their tariff rates phased in a period of generally 5-10 years.

\vspace{-10pt}
\section{Data}
\label{sec:Data}
Data for this project comes from combining several easily accessible databases, allowing
for straightforward replication of my results.


For tariff data in 2002, 
I use the \citet{wtotariff} Tariff Analysis database, which provides tariff rates 
at the Harmonized System (HS) six digit level. To compute tariffs in 2013, 
I employ direct text from the CAFTA-DR bill, provided online by the 
\citet{ustraderep} at the HS eight digit level, as well as the source above. 



I use several sources to estimate the share of economic activity in a given municipality
in the Dominican Republic. 

Information on the number of companies by size in a given industry at the 
municipality level in the Dominican Republic is provided by \citet{one} ONE for 2010
\footnote{Note -- this source used to be accessible online, but when I recently went back
the site threw an error when attempting to download the data.}. I then combine
this with Integrated Public Use Microdata Series International
% Cite this
(IPUMS) data 

%To aggregate the Harmonized System data to a SITC code, I use simple averages, 
%as weighting by trade volume is impossible at this point. 
% How am I aggregating tariffs again?


\vspace{-10pt}
\section{Exogeneity of Tariff Decreases}
\label{sec:Exogeneity}
Before discussing the impact of tariff changes upon regional and national outcomes in the Dominican 
Republic, it is first necessary to comment upon the political economy of tariff negotiations. 
\citet{brock1978}, as well as others, have observed the potential for special interest lobbies within 
a country to have a significant impact upon policymakers and their decisions, particularly when 
deciding which trade barriers to remove and which to leave in place (or, in rarer cases, increase). 
One consequence of this is that tariff rates and tariff rate reductions could be determined by 
endogenous factors also affecting regional labor markets. If this is the case, then 
estimates for the effect of tariffs on either of these two factors will suffer from omitted variable 
bias. Therefore, I discuss the qualitative and quantitative evidence available on the potential 
exogeneity of tariff rate reductions as a result of CAFTA-DR. Although each country negotiated tariff 
rate reductions individually with the United States, as previously stated, the goals of the agreement 
were to lower tariffs on all incoming goods. If the United States exercised significant power in the 
negotiating process to lower tariffs uniformly, it is possible that reductions due to CAFTA-DR can be 
viewed as politically exogenous. Similarly, if policymakers in the Dominican Republic were interested 
in lowering trade barriers uniformly, they may have been immune to political pressures from local 
industries and firms. Although I have not found specific evidence of policymakers stated interests, 
there is sufficient evidence to believe something of the sort occurred. First, 2006 D.R. tariffs 
resemblance tariff barriers in place in 1996 very closely. Therefore, as many of the tariff barriers 
were set decades ago, it is less likely that their level is reflective of modern industrial and 
economic forces, and more likely to be due to institutional constraints and political difficulties in 
lowering tariffs without an intervention from another country (the United States, in this case). 
Second, I compare the relationship between pre-CAFTA tariff levels and the amount of tariff reduction 
in figures %\ref{fig:Figure1} and \ref{fig:Figure2}
. If policymakers had the interest in lowering tariffs uniformly, one would expect 
to see larger tariff reductions on products that had initially high protection levels. Indeed, for 
average tariffs on SITC 3 digit product codes, we see a nearly linear relationship between the 
initial amount of protection for a good and the amount of tariff decrease. These results alone 
suggest that Dominican tariff protection was lowered more or less uniformly by CAFTA-DR.


\newpage
%\section{Works Cited}
\singlespacing
\bibliography{mylib.bib}
\bibliographystyle{plainnat}

%\newpage
\appendix
\singlespacing

\section{Tables}
\label{sec:Tables}
\fontsize{10pt}{12pt}\selectfont


%\newpage
\section{Figures}
\label{sec:Figures}

%\section{Proofs}

%\section{Data Notes}

\end{document}
