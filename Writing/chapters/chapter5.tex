\chapter{Data}
\label{sec:Data}

\newthought{Data for this project} comes from combining several easily accessible databases, allowing
for straightforward replication of my results\footnote{Additionally, replication code for this 
paper is available at \url{https://github.com/jaysayre/cafta-dr}}.
To measure the extent of trade liberalization, I estimate the level of trade barriers in the
Dominican Republic in 2002 and 2013. These years correspond with the years that household
survey data is available for the Dominican Republic. The 2013 household survey results were collected 
between July and October of 2013; this implies
that the later portion of my panel dataset was collected than six years after CAFTA-DR was implemented 
in March 2007. In theory, this should hopefully be long enough for local labor markets to adjust 
to new changes in tariff barriers.
Between 2002 and 2006, Dominican Republic duties on American goods 
remained largely constant until the passage of CAFTA (Figure \ref{fig:Graph3}), so the 
difference in duties between 2002 and 2013 is primarily a result of the free trade agreement.

For tariff data in 2002, 
I use the \citet{wtotariff} Tariff Analysis database, which provides tariff information 
at the Harmonized System (HS) six digit level\footnote{The same source
also provides harmonized system two digit level information,
which I use for Figure \ref{fig:Graph3}}. To compute tariffs in 2013, 
I employ direct text from the CAFTA-DR bill, provided online by the 
\citet{ustraderep} at the HS eight digit level. The treaty provides
information on the base tariff rates of each eight digit good, and the tariff phase out
scheme for each good (Appendix \ref{fig:Appendix1}).
Using information on each phase out scheme, I calculate the estimated tariff
for each good in 2013. To combine these sources, I aggregate
the CAFTA-DR tariff information to the six digit level using an unweighted average, 
since, to my knowledge, trade volume statistics are not readily available at the HS eight digit level.
I then need to match up industrial products to their respective occupations to calculate
the estimated input tariff that a given occupation faces. 
To do this, I use a standard product to occupation concordance table\footnote{Found at 
\href{http://wits.worldbank.org/product_concordance.html}{World Integrated Trade Solution (WITS)}, 
provided by the World Bank.} to convert duties from the Harmonized System 6 digit level to the 
International Standard Industrial Classification (ISIC) four digit 
level\footnote{Specifically, a table from HS 1996 TO ISIC Rev. 3.1.}, which I then take 
simple averages of to aggregate to the ISIC two digit level.
For obvious reasons\footnote{For example, it is unclear what effect various inputs tariffs have
upon occupations in, say, the service sector.}, the concordance table matches product information
for only some of the occupation codes found in my survey data. Following the convention employed 
by the literature, I set the corresponding tariff faced by these occupations to zero.

For the survey data previously alluded to, I use two sources. 
Dominican Republic household survey data for 2002 comes from the 
Integrated Public Use Microdata Series International
(IPUMS) database, produced by the \citet{ipumsi} and conducted by the
\citet{one} (ONE). The IPUMS data provides information on 
survey respondents' income (measured as monthly total income in
2002 Dominican pesos), municipality of residence, and occupation,
provided at the ISIC two digit level, in addition to a host of other characteristics.
%More information on geographic displacement in appendix?
Household level data for 2013 comes from the 
Demographic and Health Surveys (DHS) Program, and is produced by \citet{dhs}.
Although the DHS dataset mostly provides information on the health 
characteristics of survey respondents, it also provides 
information on a respondent's occupation, place of residence,
and occupational income (provided in weekly 2013 Dominican pesos), in addition to other factors. 
DHS occupational information is provided without reference to any existing occupational/industrial 
classification system, so I convert it manually to
ISIC two digit codes (see Appendix \ref{fig:Appendix2}).
Although both sources provide weekly income data, I am primarily interested in data
on the average wage rate of workers in a given occupation, so I divide this data by the average 
amount of hours worked per week by occupation (see Appendices \ref{fig:Appendix3} and 
\ref{fig:Appendix4}). %More information on this construction in appendix?

\textit{Municipality level regressions}

For my estimating equations at the municipal level,
I use several sources to estimate the share of economic activity in a given municipality.
Information on the number of firms by size (measured in terms of
number of workers employed) in a given industry at the 
municipality level in the Dominican Republic is provided by the 
Directory of Companies and Establishments (\textit{Directorio de Empresas y Establecimientos})
provided by ONE for 2010\footnote{Available online at \href{http://www.one.gov.do/recursos-automatizados/
323/directorio-de-empresas-y-establecimientos/}{ONE}. Note: this page, in my experience,
only works sporadically.}. This information is provided
at the International Standard Industrial Classification four digit level, which I sum
up to the ISIC two digit level by municipality and number of workers employed.
I then combine this plant level data with IPUMS survey data from 2002 and 2010 to provide 
clearer estimates of the number of workers employed in each industry. Recalling that occupational data
for IPUMS is provided at the ISIC two digit level, I use this to compute the estimated
share of industrial activity per municipality for both 2002 and 2010.
% More details here?
Next, I merge each ISIC occupation code with the four digit ISIC duties computed above, 
and then estimate the average level of tariff in a municipality using a weighted average
based upon the estimated number of workers in a given occupation in that 
municipality\footnote{i.e. Average Tariff$_m = \bar{t}_m =
\dfrac{\displaystyle \sum_{\text{All occupations}_m} |workers_{o,m}| \cdot t_o}
{\displaystyle \sum_{\text{All occupations}_m} |workers_{o,m}|}$, where $o$ is the 
occupation, $m$ is a municipality, and $t$ is the tariff rate}. Finally, I aggregate
the wage rate for workers listed as currently employed by the private sector who 
have occupations for in the 2013 DHS 
and 2002 IPUMS survey data to the municipality level and merge
this to the average tariff data. 
To accurately compare changes in wages, I convert 2002 
and 2013 Dominican monthly wages in pesos to 2013 US Dollars using the nominal exchange rate.

\textit{Occupational level regressions}

Next, for my estimating equations at the occupational 
and municipal level, I aggregate the wage rate for workers listed as currently employed 
by the private sector in the the 2013 DHS and 2002 IPUMS survey data to the 
municipal and occupational (ISIC two digit) level, and as above, 
convert this income data to 2013 United States Dollars. 
This is then merged with the ISIC tariff data at the national level, 
which I aggregate to the ISIC two digit level using an unweighted average. 