\chapter{Tables}
\label{sec:Tables}
\fontsize{10pt}{12pt}\selectfont

\begin{figure}[H]
\begin{center}
National statistics, 2002 and 2013

\begin{tabular}{rrrrrrr}
\toprule
 year &      duty &    income &  employment &       edu &           pop &         empop \\
\midrule
 2002 &  9.090764 &  1.550120 &  147.237795 &  9.140800 &  55242.200000 &  20481.251613 \\
 2010 &  1.751566 &       NaN &  190.752946 &  8.977756 &  60937.296774 &  19544.870968 \\
 2013 &  1.542419 &  1.469766 &         NaN &  8.177072 &           NaN &           NaN \\
\bottomrule
\end{tabular}

\caption{\label{fig:Summary 1}}
\end{center}
Income measured
as average hourly wage rate for respondents in terms of 2013 USD, 
duty is average municipality level tariff (see data
section for construction), edu is average years of education of survey respondents. 
2013 education data comes from the DHS, whereas 2002 and 2010 education data comes
from IPUMS International. pop is the average population of each municipality,
and empop is the average employed population in a given municpality.
\end{figure}

\begin{figure}[H]
\begin{center}
Summary statistics at the provincial level, 2002 and 2013
\begin{tabular}{lrrrrr}
\toprule
               Province &   duty2002 &  duty2013 &    edu2002 &    edu2013 &  pop2002 \\
\midrule
      Distrito Nacional &   8.240694 &  1.399397 &  11.652920 &  10.053191 &   910076 \\
                   Azua &   8.183071 &  1.753554 &   6.673492 &   6.757792 &   208857 \\
                Baoruco &   8.322165 &  1.164303 &   8.272419 &   6.676802 &    91480 \\
               Barahona &   9.974991 &  1.152558 &   7.868843 &   8.018119 &   179239 \\
                Dajabon &   7.928187 &  1.134944 &   8.369262 &   7.914828 &    62046 \\
                 Duarte &   9.021363 &  2.150494 &   7.696067 &   8.264359 &   283805 \\
             Elias Pina &   8.318384 &  0.981835 &   7.842340 &   4.170476 &    63879 \\
               El Seibo &  11.631243 &  2.669896 &   6.219797 &   4.947990 &    89261 \\
              Espaillat &   8.223223 &  1.654410 &   7.977944 &   8.108730 &   225091 \\
          Independencia &   9.533839 &  1.156563 &   7.978710 &   6.521484 &    50833 \\
          La Altagracia &   9.982569 &  1.786783 &   6.806431 &   6.957379 &   182020 \\
              La Romana &   8.413022 &  1.294545 &   8.513383 &   7.738636 &   219812 \\
                La Vega &  10.979687 &  2.487073 &   7.293923 &   7.205999 &   385101 \\
 Maria Trinidad Sanchez &   8.741044 &  1.645261 &   7.940342 &   8.760175 &   135727 \\
           Monte Cristi &   9.753788 &  2.061071 &   7.011730 &   6.029448 &   111014 \\
             Pedernales &  10.943896 &  2.475274 &   7.120327 &   5.285714 &    21207 \\
                Peravia &   9.504346 &  1.125880 &   7.032138 &   7.596199 &   169865 \\
           Puerto Plata &   8.656353 &  1.606013 &   7.895205 &   9.336413 &   312706 \\
       Hermanas Mirabal &   9.787518 &  1.182684 &   8.771925 &   7.679012 &    96356 \\
                 Samana &   6.817380 &  0.666834 &   7.455959 &   7.902262 &    91875 \\
          San Cristobal &   7.379443 &  1.433770 &   8.032860 &   7.803579 &   532880 \\
               San Juan &   9.465810 &  1.388402 &   7.987854 &   5.739059 &   241105 \\
   San Pedro De Macoris &   9.755582 &  1.132217 &   7.313653 &   7.423306 &   301744 \\
        Sanchez Ramirez &   9.407157 &  1.497831 &   8.391827 &   7.505265 &   151179 \\
               Santiago &   8.647065 &  1.775339 &   7.471126 &   7.336411 &   908250 \\
     Santiago Rodriguez &   8.681414 &  1.505634 &   8.338438 &   8.020489 &    59629 \\
               Valverde &  10.690356 &  2.246039 &   7.437436 &   7.623996 &   158293 \\
         Monsenor Nouel &   7.752975 &  0.811129 &   8.657205 &   9.550799 &   167618 \\
            Monte Plata &   9.026874 &  1.131544 &   7.453535 &   6.097778 &   180376 \\
             Hato Mayor &  11.226879 &  1.671579 &   6.722243 &   7.027014 &    87631 \\
       San Jose De Ocoa &  12.703298 &  3.480024 &   5.852583 &   7.000490 &    62368 \\
          Santo Domingo &   8.129143 &  0.878310 &   9.181434 &   9.282954 &  1821218 \\
\bottomrule
\end{tabular}

\caption{\label{fig:Summary 2}}
\end{center}
duty is average provincial level tariff (see data
section for construction), edu is average years of education of survey respondents. 
2013 education data comes from the DHS, 2002 education data comes
from IPUMS International. pop is the total population of a province.
\end{figure}

\begin{figure}[H]
\begin{center}
Summary statistics at the provincial level, 2002 and 2013 (continued)
\begin{tabular}{lrrrrr}
\toprule
               Province &  emprate2002 &  emprate2010 &  income2002 &  income2013 &  pop2010 \\
\midrule
      Distrito Nacional &     0.680864 &     0.558596 &    2.610931 &    2.402139 &   965040 \\
                   Azua &     0.536060 &     0.418833 &    0.876307 &    1.059907 &   214311 \\
                Baoruco &     0.455652 &     0.350401 &    1.018178 &    1.250049 &    97313 \\
               Barahona &     0.512306 &     0.387926 &    1.713665 &    1.341210 &   187105 \\
                Dajabon &     0.589380 &     0.457170 &    1.195142 &    1.164268 &    63955 \\
                 Duarte &     0.576135 &     0.471745 &    0.769377 &    1.314396 &   289574 \\
             Elias Pina &     0.471521 &     0.292214 &    1.073069 &    0.835818 &    63029 \\
               El Seibo &     0.590494 &     0.465563 &    0.728377 &    0.965732 &    87680 \\
              Espaillat &     0.613326 &     0.534692 &    1.224801 &    1.287659 &   231938 \\
          Independencia &     0.454935 &     0.376758 &    1.290767 &    1.233975 &    52589 \\
          La Altagracia &     0.669881 &     0.637948 &    1.240365 &    1.129366 &   273210 \\
              La Romana &     0.679967 &     0.532364 &    1.200504 &    1.519451 &   245433 \\
                La Vega &     0.596732 &     0.503689 &    0.861513 &    1.225923 &   394205 \\
 Maria Trinidad Sanchez &     0.553301 &     0.481833 &    0.871600 &    1.354070 &   140925 \\
           Monte Cristi &     0.589138 &     0.474859 &    0.789282 &    1.290063 &   109607 \\
             Pedernales &     0.627069 &     0.549792 &    1.304174 &    1.004340 &    31587 \\
                Peravia &     0.596822 &     0.459732 &    0.764039 &    1.169369 &   184344 \\
           Puerto Plata &     0.607972 &     0.490467 &    0.902419 &    1.039885 &   321597 \\
       Hermanas Mirabal &     0.581598 &     0.453808 &    0.845411 &    1.175354 &    92193 \\
                 Samana &     0.547195 &     0.461109 &    0.888578 &    1.523580 &   101494 \\
          San Cristobal &     0.614950 &     0.493485 &    1.035849 &    1.194254 &   569930 \\
               San Juan &     0.490393 &     0.422654 &    1.078948 &    0.998734 &   232333 \\
   San Pedro De Macoris &     0.647417 &     0.485386 &    0.861584 &    0.982007 &   290458 \\
        Sanchez Ramirez &     0.531247 &     0.438957 &    0.880525 &    1.488925 &   151392 \\
               Santiago &     0.653234 &     0.533027 &    0.903622 &    1.312270 &   963422 \\
     Santiago Rodriguez &     0.525596 &     0.406506 &    0.913203 &    1.277552 &    57476 \\
               Valverde &     0.657002 &     0.536210 &    0.898341 &    1.263166 &   163030 \\
         Monsenor Nouel &     0.559126 &     0.470477 &    1.047613 &    1.437918 &   165224 \\
            Monte Plata &     0.579148 &     0.450097 &    1.200484 &    1.198388 &   185956 \\
             Hato Mayor &     0.591947 &     0.452315 &    0.852228 &    1.642278 &    85017 \\
       San Jose De Ocoa &     0.612520 &     0.517947 &    0.834089 &    1.241903 &    59544 \\
          Santo Domingo &     0.634053 &     0.530573 &    1.500772 &    1.330638 &  2374370 \\
\bottomrule
\end{tabular}


\vspace{15pt} \textbf{Figure \ref{fig:Summary 2}}
\end{center}
Income measured
as average hourly wage rate in terms of 2013 USD within a province.
emprate is the employment rate, calculated as the total employed work force of a province
over the total provincial population of working age.
pop is the total population of a province.
\end{figure}

\begin{landscape}
\begin{figure}[H]
\begin{center}
Municipality level effect of tariff changes on migration and size of work force

\begin{tabular}{l c c c c c c }
\hline
 & Log Change in & Population & Log Change in & Employed Workers & Log Change in & Employment Rate \\
\hline
Intercept & $-0.00$  & $0.04$   & $-0.15^{***}$ & $-0.13^{***}$ & $-0.20^{***}$ & $-0.20^{***}$ \\
            & $(0.04)$ & $(0.04)$ & $(0.04)$      & $(0.04)$      & $(0.03)$      & $(0.02)$      \\
Change in tariff     & $-0.03$  & $-0.02$  & $-0.01$       & $-0.03$       & $0.03$        & $-0.01$       \\
            & $(0.05)$ & $(0.04)$ & $(0.06)$      & $(0.04)$      & $(0.04)$      & $(0.02)$      \\
\hline
FE          &                &                &                &                &                &                \\ 
\hline
$R^2$       & 0.01     & 0.42     & 0.00          & 0.42          & 0.01          & 0.45          \\
N           & 155      & 155      & 155           & 155           & 155           & 155           \\
\hline
\multicolumn{7}{l}{\scriptsize{$^{***}p<0.01$, $^{**}p<0.05$, $^*p<0.1$, Clustered-robust standard errors in parentheses}}
\end{tabular}

\caption{\label{fig:Table3}}
\end{center}
Observations are municipalities in the Dominican Republic (second administrative level),
and tariffs are averages of estimated tariffs for ISIC 2 digit occupational codes,
weighted by the number of workers in a given occupation in that municipality 
(see section \ref{sec:Data} for more details). Standard errors are clustered at the province level.
Columns 2, 4, and 6 have province level fixed effects.
``pop'' is the estimated population in a given municipality, ``empop'' is the size of
the work force employed by the private or public sector (excludes self-employment).
Employment rate is calculated as the total employed work force of a municipality
over the total municipal population of working age.
\end{figure}


\begin{figure}[H]
\begin{center}
Municipality level effect of tariff changes on change in log wage rate from 2002 to 2013 (measured
in 2013 USD)

\begin{tabular}{l c c c c c c c c }
\hline
 & (1) & (2) & (3) & (4) & (5) & (6) & (7) & (8) \\
\hline
Intercept     & $0.45^{***}$ & $0.08$   & $1.07^{***}$  & $0.68$        & $0.27$   & $-0.00$  & $0.56$       & $-0.58$       \\
                & $(0.16)$     & $(0.14)$ & $(0.26)$      & $(0.47)$      & $(0.20)$ & $(0.44)$ & $(0.70)$     & $(0.97)$      \\
$\Delta \log(t+1)$         & $0.29^{*}$   & $0.18$   & $0.22$        & $0.27$        & $0.20$   & $-0.08$  & $0.17$       & $-0.82$       \\
                & $(0.16)$     & $(0.15)$ & $(0.17)$      & $(0.17)$      & $(0.16)$ & $(0.42)$ & $(0.38)$     & $(0.71)$      \\
edu02           &              &          & $-0.12^{***}$ & $-0.16^{***}$ &          &          & $-0.11$      & $-0.14^{***}$ \\
                &              &          & $(0.04)$      & $(0.05)$      &          &          & $(0.19)$     & $(0.05)$      \\
edu10           &              &          & $0.04$        & $0.04$        &          &          & $-0.00$      & $0.22$        \\
                &              &          & $(0.04)$      & $(0.04)$      &          &          & $(0.18)$     & $(0.14)$      \\
edu13           &              &          &               & $0.10^{***}$  &          &          & $0.10^{***}$ & $0.10^{***}$  \\
                &              &          &               & $(0.02)$      &          &          & $(0.02)$     & $(0.02)$      \\
chngedu         &              &          &               &               & $0.07$   & $-0.06$  &              &               \\
                &              &          &               &               & $(0.04)$ & $(0.20)$ &              &               \\
$\Delta \log(t+1) \times$chngedu &              &          &               &               &          & $-0.14$  & $-0.05$      &               \\
                &              &          &               &               &          & $(0.19)$ & $(0.18)$     &               \\
$\Delta \log(t+1) \times$edu10   &              &          &               &               &          &          &              & $0.19$        \\
                &              &          &               &               &          &          &              & $(0.12)$      \\
\hline
Province FE & No & Yes & No & Yes & Yes & Yes & Yes & Yes  \\ 
\hline
$R^2$           & 0.03         & 0.43     & 0.10          & 0.60          & 0.45     & 0.45     & 0.60         & 0.61          \\
N               & 130          & 130      & 130           & 130           & 130      & 130      & 130          & 130           \\
\hline
\multicolumn{9}{l}{\scriptsize{$^{***}p<0.01$, $^{**}p<0.05$, $^*p<0.1$, Clustered-robust standard errors in parentheses}}
\end{tabular}

\caption{\label{fig:Table1}}
\end{center}
Observations are municipalities in the Dominican Republic (second administrative level),
and tariffs are averages of estimated tariffs for ISIC 2 digit occupational codes,
weighted by the number of workers in a given occupation in that municipality 
(see section \ref{sec:Data} for more details). Standard errors are clustered at the province level.
edu is average years of education of survey respondents in a municipality in a given year,
and chngedu is the change of average years of education from 2002-2010.
\end{figure}

\begin{figure}[H]
\begin{center}
Effect of tariff changes on change in log wage rate from 2002 to 2013 (measured in 2013 USD)

\begin{tabular}{l c c c c c c c c c }
\hline
 & (1) & (2) & (3) & (4) & (5) & (6) & (7) & (8) & (9) \\
\hline
Intercept          & $0.38^{***}$ & $0.25^{***}$ & $0.32^{***}$ & $0.27^{***}$ & $0.38^{***}$ & $0.19^{**}$  & $0.21$       & $-0.11$      & $0.01$        \\
                     & $(0.06)$     & $(0.04)$     & $(0.07)$     & $(0.05)$     & $(0.07)$     & $(0.09)$     & $(0.16)$     & $(0.13)$     & $(0.11)$      \\
Change in tariff              & $0.49^{***}$ & $0.40^{***}$ & $0.45^{***}$ & $0.38^{***}$ & $0.45^{***}$ & $0.40^{***}$ & $0.41^{***}$ & $0.44^{***}$ & $0.45^{***}$  \\
                     & $(0.05)$     & $(0.04)$     & $(0.05)$     & $(0.04)$     & $(0.05)$     & $(0.06)$     & $(0.11)$     & $(0.06)$     & $(0.05)$      \\
nontraded            &              &              &              &              & $-0.16^{*}$  &              &              &              & $-0.17^{*}$   \\
                     &              &              &              &              & $(0.09)$     &              &              &              & $(0.09)$      \\
firmconc02           &              &              &              &              &              & $1.39^{**}$  & $1.18$       &              &               \\
                     &              &              &              &              &              & $(0.60)$     & $(1.56)$     &              &               \\
Change in tariff:firmconc02   &              &              &              &              &              &              & $-0.16$      &              &               \\
                     &              &              &              &              &              &              & $(1.07)$     &              &               \\
numworkers02         &              &              &              &              &              &              &              & $0.00$       &               \\
                     &              &              &              &              &              &              &              & $(0.00)$     &               \\
Change in tariff:numworkers02 &              &              &              &              &              &              &              & $-0.00$      &               \\
                     &              &              &              &              &              &              &              & $(0.00)$     &               \\
numworkers10         &              &              &              &              &              &              &              &              & $0.00^{*}$    \\
                     &              &              &              &              &              &              &              &              & $(0.00)$      \\
Change in tariff:numworkers10 &              &              &              &              &              &              &              &              & $-0.00^{***}$ \\
                     &              &              &              &              &              &              &              &              & $(0.00)$      \\
\hline
FE          &                &                &                &                &                &                &                &                &                \\ 
\hline
$R^2$                & 0.06         & 0.07         & 0.27         & 0.24         & 0.24         & 0.28         & 0.28         & 0.28         & 0.25          \\
N                    & 980          & 1184         & 980          & 1184         & 1184         & 980          & 980          & 980          & 1184          \\
\hline
\multicolumn{10}{l}{\scriptsize{$^{***}p<0.01$, $^{**}p<0.05$, $^*p<0.1$, Clustered-robust standard errors in parentheses}}
\end{tabular}

\caption{\label{fig:Table2}}
\end{center}
Observations are each occupation within a municipality.
Tariff rates are estimated duties for each ISIC 2 digit occupational code,
using a concordance table from HS1996 to ISIC Rev.3.1. Standard errors are clustered at the 
municipality level. numworkers is the estimated
number of workers within a municipality in the given occupation. occupationconc is the measure of
the concentration of a ISIC 2 digit occupation within a municipality. nontraded is a binary variable
taking the value one if an occupation has no corresponding tariff information, and so the change in
tariff rate has been set to zero.
\end{figure}

\begin{figure}[H]
\begin{center}
Heterogeneity: Effect of tariff changes on change in log wage rate from 2002 to 2013, high skill sample

\begin{tabular}{l c c c c c c c c c }
\hline
 & (1) & (2) & (3) & (4) & (5) & (6) & (7) & (8) & (9) \\
\hline
Intercept          & $0.30^{***}$ & $0.37^{***}$ & $0.40^{***}$ & $3.17^{***}$  & $-2.30^{***}$ & $0.37^{***}$ & $-2.72^{***}$ & $2.81^{***}$  & $-2.80^{***}$ \\
                     & $(0.05)$     & $(0.10)$     & $(0.05)$     & $(0.31)$      & $(0.36)$      & $(0.11)$     & $(0.41)$      & $(0.33)$      & $(0.37)$      \\
Change in tariff              & $0.44^{***}$ & $0.48^{***}$ & $0.36^{***}$ & $0.17^{*}$    & $0.29^{***}$  & $0.37^{***}$ & $0.28^{***}$  & $0.16^{**}$   & $0.29^{***}$  \\
                     & $(0.05)$     & $(0.07)$     & $(0.06)$     & $(0.09)$      & $(0.09)$      & $(0.09)$     & $(0.09)$      & $(0.07)$      & $(0.09)$      \\
nontraded            &              & $-0.11$      &              & $-0.04$       & $-0.08$       & $-0.05$      & $1.58^{**}$   &               & $-0.11$       \\
                     &              & $(0.12)$     &              & $(0.13)$      & $(0.13)$      & $(0.14)$     & $(0.70)$      &               & $(0.13)$      \\
edu02                &              &              &              & $-0.22^{***}$ &               &              &               & $-0.22^{***}$ &               \\
                     &              &              &              & $(0.03)$      &               &              &               & $(0.02)$      &               \\
edu13                &              &              &              &               & $0.20^{***}$  &              & $0.24^{***}$  &               & $0.20^{***}$  \\
                     &              &              &              &               & $(0.03)$      &              & $(0.03)$      &               & $(0.03)$      \\
Change in employment              &              &              &              &               &               & $0.00$       &               &               &               \\
                     &              &              &              &               &               & $(0.00)$     &               &               &               \\
edu13:nontraded      &              &              &              &               &               &              & $-0.13^{**}$  &               &               \\
                     &              &              &              &               &               &              & $(0.06)$      &               &               \\
numworkers02         &              &              &              &               &               &              &               & $0.00$        &               \\
                     &              &              &              &               &               &              &               & $(0.00)$      &               \\
Change in tariff:numworkers02 &              &              &              &               &               &              &               & $-0.00^{***}$ &               \\
                     &              &              &              &               &               &              &               & $(0.00)$      &               \\
numworkers10         &              &              &              &               &               &              &               &               & $0.00^{**}$   \\
                     &              &              &              &               &               &              &               &               & $(0.00)$      \\
Change in tariff:numworkers10 &              &              &              &               &               &              &               &               & $-0.00^{***}$ \\
                     &              &              &              &               &               &              &               &               & $(0.00)$      \\
\hline
FE          &                &                &                &                &                &                &                &                &                \\ 
\hline
$R^2$                & 0.08         & 0.08         & 0.29         & 0.41          & 0.37          & 0.29         & 0.38          & 0.41          & 0.38          \\
N                    & 765          & 765          & 765          & 765           & 765           & 765          & 765           & 765           & 765           \\
\hline
\multicolumn{10}{l}{\scriptsize{$^{***}p<0.01$, $^{**}p<0.05$, $^*p<0.1$, Clustered-robust standard errors in parentheses}}
\end{tabular}

\caption{\label{fig:Table4}}
\end{center}
\small{Observations are each occupation within a municipality.
Tariff rates are estimated duties for each ISIC 2 digit occupational code,
using a concordance table from HS1996 to ISIC Rev.3.1. Standard errors are clustered at the 
municipality level. The high skill sample only 
contains survey respondents with 9 or more years of education. numworkers is the estimated
number of workers within a municipality in the given occupation, and change in employment
is the change in numworkers from 2002 to 2010. edu is average years of education 
of workers in an occupation within a municipality in a given year. nontraded is a binary variable
taking the value one if an occupation has no corresponding tariff information (i.e. it produces goods
that are not traded) and so the change in
tariff rate has been set to zero.}
\end{figure}

\begin{figure}[H]
\begin{center}
Heterogeneity: Effect of tariff changes on change in log wage rate from 2002 to 2013, low skill sample

\begin{tabular}{l c c c c c c c c c }
\hline
 & (1) & (2) & (3) & (4) & (5) & (6) & (7) & (8) & (9) \\
\hline
Intercept          & $0.37^{***}$ & $0.57^{***}$  & $0.30^{***}$ & $0.82^{***}$  & $0.20$       & $0.46^{***}$ & $0.21$       & $0.48^{***}$  & $-0.04$      \\
                     & $(0.05)$     & $(0.08)$      & $(0.06)$     & $(0.15)$      & $(0.18)$     & $(0.10)$     & $(0.20)$     & $(0.16)$      & $(0.20)$     \\
$\Delta\log(t+1)$    & $0.15^{***}$ & $0.29^{***}$  & $0.13^{**}$  & $0.15^{*}$    & $0.23^{***}$ & $0.24^{***}$ & $0.23^{***}$ & $0.06$        & $0.23^{***}$ \\
                     & $(0.05)$     & $(0.07)$      & $(0.06)$     & $(0.08)$      & $(0.07)$     & $(0.07)$     & $(0.07)$     & $(0.06)$      & $(0.07)$     \\
nontraded            &              & $-0.34^{***}$ &              & $-0.18$       & $-0.26^{**}$ & $-0.29^{**}$ & $-0.31$      &               & $-0.29^{**}$ \\
                     &              & $(0.10)$      &              & $(0.12)$      & $(0.11)$     & $(0.13)$     & $(0.31)$     &               & $(0.12)$     \\
edu02                &              &               &              & $-0.09^{***}$ &              &              &              & $-0.10^{***}$ &              \\
                     &              &               &              & $(0.03)$      &              &              &              & $(0.03)$      &              \\
edu13                &              &               &              &               & $0.04$       &              & $0.04$       &               & $0.04$       \\
                     &              &               &              &               & $(0.03)$     &              & $(0.03)$     &               & $(0.03)$     \\
$\Delta$numworkers   &              &               &              &               &              & $0.00$       &              &               &              \\
                     &              &               &              &               &              & $(0.00)$     &              &               &              \\
edu13$\times$nontraded      &              &               &              &               &              &              & $0.01$       &               &              \\
                     &              &               &              &               &              &              & $(0.07)$     &               &              \\
numworkers02         &              &               &              &               &              &              &              & $0.00^{***}$  &              \\
                     &              &               &              &               &              &              &              & $(0.00)$      &              \\
$\Delta\log(t+1)\times$numworkers02 &              &               &              &               &              &              &              & $-0.00$       &              \\
                     &              &               &              &               &              &              &              & $(0.00)$      &              \\
numworkers10         &              &               &              &               &              &              &              &               & $0.00^{***}$ \\
                     &              &               &              &               &              &              &              &               & $(0.00)$     \\
$\Delta\log(t+1)\times$numworkers10 &              &               &              &               &              &              &              &               & $-0.00$      \\
                     &              &               &              &               &              &              &              &               & $(0.00)$     \\
\hline
Municipality FE  &No&No&Yes&Yes&Yes&Yes&Yes&Yes&Yes\\ 
\hline
$R^2$                & 0.01         & 0.02          & 0.24         & 0.27          & 0.25         & 0.25         & 0.25         & 0.27          & 0.26         \\
N                    & 734          & 734           & 734          & 734           & 734          & 734          & 734          & 734           & 734          \\
\hline
\multicolumn{10}{l}{\scriptsize{$^{***}p<0.01$, $^{**}p<0.05$, $^*p<0.1$, Clustered-robust standard errors in parentheses}}
\end{tabular}

\caption{\label{fig:Table5}}
\end{center}
\small{Observations are each occupation within a municipality. 
Tariff rates are estimated duties for each ISIC 2 digit occupational code,
using a concordance table from HS1996 to ISIC Rev.3.1. Standard errors are clustered at the 
municipality level. The low skill sample only 
contains survey respondents with 0-8 years of education. numworkers is the estimated
number of workers within a municipality in the given occupation, and change in employment
is the change in numworkers from 2002 to 2010. edu is average years of education 
of workers in an occupation within a municipality in a given year. nontraded is a binary variable
taking the value one if an occupation has no corresponding tariff information and so the change in
tariff rate has been set to zero.}
\end{figure}
\end{landscape}