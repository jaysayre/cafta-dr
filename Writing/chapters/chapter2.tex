\chapter{Context}
\label{sec:Context}

\newthought{Over the last several decades}, trade barriers have fallen substantially, and agreements promoting
free trade between countries have proliferated. One such agreement, the Central American Free Trade 
Agreement aimed to lower trade barriers between Central American countries and 
the United States. One of the explicit aims of CAFTA was to phase out tariffs on U.S. 
imports into member Central American countries, or ``progressively eliminate customs duties on 
originating goods'' \citep{ustraderep}.

In 2003, negotiations began on the Central American Free Trade Agreement, with Costa Rica, El 
Salvador, Guatemala, Honduras, Nicaragua, and the United States taking part in the discussions. The 
Dominican Republic joined the negotiations in early January, retitling the agreement the Dominican 
Republic-Central America Free Trade Agreement. U.S. President George W. 
Bush signed CAFTA into law in 2005, but it took another two years for the 
Dominican Republic to fully implement the agreement, which it did on March 1, 2007.
Based upon World Trade Organization data,
I find CAFTA-DR reduced average Dominican 
Republic (D.R.) tariff rates on imports from member countries from 12.06\% to 2.73\% from 2006 to 
2007\footnote{Using a import-weighted average
of tariff rates for Harmonized System two-digit product codes, 
the unweighted average decrease is larger.}. The magnitude of this decrease on imported American
goods is similar to the size of the decrease in Mexican tariffs on American goods as a result of the
North American Free Trade Agreement \citep{goldberg}. 
In recent history, other countries such as Brazil reduced tariff barriers more drastically
\citep{kovak}; however, this is still economically meaningful, especially since the United States 
is the largest trading partner of the Dominican Republic, composing 
38.6 percent of  total imports into the D.R in 2013 \citep{wtocountry}\footnote{The 
United States also receives 56 percent of total exports from the Dominican Republic.}.
% Percentage of trade from CAFTA member countries?

CAFTA established a moratorium on creating new tariff lines or raising customs 
duties between the parties involved, and explicitly defined a time table for each good to have its 
tariffs reduced. Of the goods that the Dominican Republic had formerly placed tariff barriers upon,
70.4\% of goods originating from the United States became duty free in 2007, another 6.5\%
of goods became duty free by 2012 (2013 being the year for which I have survey data for), with
the remaining 22.9\% of goods having partially reduced tariffs by 2013.
In summation, many goods were to be declared duty free initially upon implementation of the agreement, 
but many more were to have their duties phased out in a period of generally 5-10 years 
(see Figure \ref{fig:Appendix1} for more information). 

Tariffs on most products exported to the United States from Caribbean countries were 
already duty-free as part of the Caribbean Basin Economic Recovery Act (CBERA), and so CAFTA largely 
removed ad-valorem taxes on American imports imposed by Carribean countries. 
Implemented on January 1, 1984, CBERA eliminated U.S. import duties on goods (with certain
exceptions) from 20 initial Caribbean countries, including the Dominican Republic.
Although other specialized agreements to reduce import tariffs on goods originating from developing 
countries existed, such as the Generalized System of Preferences, CBERA did not have any ``graduation''
requirement for countries that became middle or high income, only that eligible  
goods must have at least 35\% of their value added within one or more beneficiary countries 
\citep{pelzman}. Indeed, based on \citeauthor{wtotariff} data, I find that simple Harmonized
System two-digit U.S. import tariff averages were 1.66 for CBERA countries in 2002 and 0.11 for 
CAFTA-DR countries in 2007. Although this represents a small decrease in export tariffs on
the Dominican Republic, the magnitude of this decrease is much smaller than the corresponding
decrease in import tariffs. Due to this structure, this allows me to examine the impact of reduced 
input tariff barriers on local Dominican import-competing producers, without having to 
examine simultaneous and large changes in output tariffs.
% Do I need a graph of export tariffs?

One relevant consideration for this study is the confounding effects of macroeconomic shocks 
taking place during my study. The economic crisis of 2008 had worldwide effects, particularly
on trading partners of the United States. Although I cannot say anything certain about the impact
of macroeconomic shocks on labor markets in the Dominican Republic during my study period, 
it is clear that labor markets entered a slump during this time. At the municipality level, employment
rates fell more or less uniformly from the period from 2002 to 2010\footnote{These are the years for
which demographic statistics are available, unfortunately this data for 2013 is not available.} (see
Figure \ref{fig:Map6}), despite broad increases in the population during that time (Figure \ref{fig:Map5}).
As corroborating evidence that labor markets experienced a slump, survey data on incomes suggests
that the wage rate (measured as 2013 USD per hour) experienced little nominal growth from 
2002 to 2013 (see Figure \ref{fig:Summary 1}).