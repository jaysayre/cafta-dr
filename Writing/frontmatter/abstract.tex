% the abstract
I examine the impact of modern day trade liberalization on the wages of workers in the Dominican
Republic. Upon implementation, the Central American Free Trade Agreement reduced nominal Dominican input tariffs
from an average of 12.06\% to 2.73\% from member countries, particularly the United States, and 
put regulations in place to remove remaining tariffs in a short time period after that. 
At the regional level, I find insignificant effects of trade reform
on wages. At the occupational level within a region, I find that a 10 percentage point decrease in 
input tariffs over the time period is associated with 4.5 percentage point lower wage growth over the period 2002 to 2013. 
Upon considering the heterogeneous effects of trade reform based upon skill levels of workers, I find 
that the wages of skilled workers experienced slower wage growth than their unskilled counterparts over 
the period, which is broadly consistent with predictions of the Heckscher-Ohlin Model for a developing
country.